\section{Data Sources}

A successful Discovery Agent relies on a variety of data sources. The following sources have been identified, ranked by their usefulness for this project.

\begin{enumerate}
    \item \textbf{Official Business Registries (CVR):} The most reliable source for basic company information.
    \item \textbf{Industry-Specific Portals \& Directories:} High-value targets for finding companies in a specific sector.
    \item \textbf{Company Websites:} The primary source for detailed manufacturing capabilities.
    \item \textbf{Google Maps/Business Listings:} Good for finding locations and basic contact info.
    \item \textbf{LinkedIn:} Useful for company size and focus.
\end{enumerate}

\subsection{Accessing the Danish Business Register (CVR)}

The \textbf{CVR (Central Business Register)} is the official source of company information in Denmark. While a free official API exists, it can be complex to use. For ease of integration, a \textbf{third-party CVR API} is the recommended approach.

\subsection{Legal Considerations: GDPR}

Web scraping in the EU is subject to the \textbf{General Data Protection Regulation (GDPR)}. To minimize legal risks, the following guidelines should be followed:

\begin{itemize}
    \item Focus on non-personal company data.
    \item Avoid collecting and storing information about specific employees.
    \item Always check the terms of service of the websites you are scraping.
    \item It is strongly advised to consult with a legal professional before starting any large-scale scraping project.
\end{itemize}

